\documentclass{article}
\usepackage{amsmath}
\usepackage{amsfonts}
\usepackage[]{algorithm2e}
\usepackage{scrextend}
\usepackage{listings}
\usepackage[dvipsnames,table,xcdraw]{xcolor}
\usepackage[top=5.0cm, bottom=5.0cm, left=3.0cm, right=3.0cm]{geometry}


\title{JEVisualizer documentation}
\author{Hannes Ihalainen}

\date{\today}
\begin{document}

  \maketitle
  \newpage
  \tableofcontents
  \newpage
    
  \section{Introduction}
    JEVisualizer is a program that creates a visualization video for music. The development started, when we wanted to add visualizations for our music on YouTube. One important goal is to develop JEVisualizer in such a way that it can be configured to create very different visualizations - only imagination should be the limit. \\ \\
    JEVisualizer is designed to be used by editing configuration files. N
  \newpage
  \section{How does it work}
    The basic idea is that JEVisualizer takes as an input any number of $tracks$, and outputs a video that visualizes those $tracks$. $A track$
    contains a number of $notes$, which get different intensities. Currently the program can extract tracks from the channels of wav-files
    (using an algorithm that detects frequencies) or from the tracks of mmp-files (Linux multimedia studio -files).
    
    For the output-video, any number of $layers$ can be configured. $A layer$ can contain a picture, any number of $drawers$ and any number 
    of $filters$.
    
    $A drawer$ visualizes some notes from some track(s).
    
    $Filters$ are used to create cool effects on layers or output-video.
    
    More about all these in section "How to config".
  \newpage
  \section{How to config}
    JEVisualizer contains a number of different config-files.
    
    First is main.config. It contains information about tracks and the name of visualizer-config file.
    
    Second is visualizer.config. It defines basic information about the output-video, like width, height, fps, filename etc. It also contains the layers as filenames.
    
    Thirdly there can be any number of layer-config-files. Each layer-config-file contains information about that layer. It can contain
    a background-image, information about when the layer is visible, a number of drawers and a number of different filters.
    
    From configuration-files it is also possible to include other configuration files, so there could also be a different config-file for some drawers or filter etc.
    
    \subsection{JEVisualizer configuration file format}
      JEVisualizer uses its own syntax in configuration files.
      
      Here are the basic rules of JEVisualizer configuration syntax:
      Configuration files are of a simple attribute-value format. Each attribute and value -pair must be on a single line (except when
      value is inside "", '' or {}, when it can contain linebreaks). Basic format is: \textbf{attribute=value}.
      
      \begin{labeling}{UPPER and lower case}
        \item [spaces and tabs]       spaces and tabs don't matter (except in values inside "", '' or \{\})
        \item [linebreaks]            linebreaks do matter, they separate different attribute-value pairs.
        \item [commenting]            // comments the end of line, multiline comments can be written between /* and */.
        \item ["", '' and {}]         Values containing spaces and linebreaks can be set inside "", '' or \{\}. "", '' and \{\} do note
                                      have any functional differences. Inside of "", '' or \{\} characters can be escaped using \-character.
                                      Note: p=\{c=\{...\}\} wouldn't work as expected, something like p=\{c=\textbackslash\{...\textbackslash\}\} should be used.
        \item [including other files] Special command: \#include="filename", includes config file "filename".
        \item [\#-character]          \#-character in attribute is the delimeter between name and key. (for some attributes there can
                                      be different keys...)
        \item [UPPER and lower case]  Parameter names will be converted to smallcase when reading file. A=b is same as a=b.
      -\# is the delimeter between parameter name and key 
      \end{labeling}
      Example: \\ \\
      % Generator: GNU source-highlight, by Lorenzo Bettini, http://www.gnu.org/software/src-highlite
\noindent
\mbox{}PARAMETER1\textcolor{BrickRed}{=}VALUE \\
\mbox{}PARAMETER2\textcolor{BrickRed}{=}\texttt{\textcolor{Red}{"{}VALUE\ WITH\ SPACES"{}}} \\
\mbox{}PARAMETER3\textcolor{BrickRed}{=}\textcolor{Red}{\{} \\
\mbox{}\ \ P\textcolor{BrickRed}{=}\textcolor{Purple}{2} \\
\mbox{}\ \ P\textcolor{BrickRed}{=}\textcolor{Purple}{2} \\
\mbox{}\ \ P3\textcolor{BrickRed}{=}\texttt{\textcolor{Red}{"{}SOME\ VALUE\ WITH\ SPACES"{}}} \\
\mbox{}\textcolor{Red}{\}} \\
\mbox{}\textbf{\textcolor{RoyalBlue}{\#include}}=\texttt{\textcolor{Red}{"{}file2.config"{}}} \\
\mbox{}PARAMETER4\#KEY1\textcolor{BrickRed}{=}\textcolor{Purple}{2.0} \\
\mbox{}PARAMETER4\#\textcolor{Purple}{6}\textcolor{BrickRed}{=}\textcolor{Purple}{9.2}
 \\ \\
      For further details see implementation in \textit{config.h} and \textit{config.cpp}.
    \subsection{main.config}
      Following pa
      \begin{labeling}{alligator}
      \item [ant] really busy all the time
      \item [chimp] likes bananas
      \item [alligator] very dangerous animal, sharp teeth, long
      muscular tail and a bit of text that is longer than one
      line and shows the alignment of text quite nicely
      \end{labeling}
        
    \subsection{Formulas}
      Most of the values that are used are given as datatype Formula (defined in formula.h and formula.cpp).\\
      The value of a formula is calculated again for every frame. Formula can be either simple constant or it can depend on many variables. 
      
      1. $CONSTANT$  \\
      simple floating-point number   \\
      2. $(CONSTANT,\ [ARRAY\_OF\_VARIABLE_COEFFICIENTS],\ MIN\_VALUE,\ MAX\_VALUE,\ SIN\_COEFFICIENT,\ IN\_SIN)$  \\
      Value of this type of formula is:  \\
      $min(max(CONSTANT + ARRAY_i*VARIABLE_i (for each i) + SIN\_COEFFICIENT * sin(IN\_SIN), MIN\_VALUE), MAX\_VALUE)$ \\
      Each parameter of is given as another formula \\
      Formula can have all parameters or n first of them, like for example $(CONSTANT,\ [ARRAY\_OF\_VARIABLE_COEFFICIENTS])$ \\

      Formulas are defined in following format:  
      Variables the formulas can use are also defined in config-files:
      $FPV\#i=name$
        where $name$ is some valid variable (see list below)
    
    
      Examples of valid formulas:
        $5$
          Formula has constant value $5$.
        $(5)$
          Formula contains a constant formula with value $5$ -> does same as formula $5$, but is a bit slower to calculated.
        $(5,\ [1])$
          Formula has value $5+FPV\#0$
        $(0,\ [],\ -1,\ 1,\ 1,\ (0,\ [1]))$
          Formula has value $sin(FPV\#0)$. (Min and max don't have effect since the value of $sin$ is always between $-1$ and $1$)



      Different variables can be defined in different scopes. Here is a list (this list is not necessarily up to date, but at least we have a list)
      In visualizer:
        frame
          index of the frame
        fps
          fps of output-video
        w
          width of output-video in pixels
        h
          height of output-video in pixels
        null or 0
          unset parameter


        

      Example:

      In file visualizer.config:
      \begin{lstlisting}
        FPV#0=frame
        LAYER#0="layer0.config"
      \end{lstlisting}

      in file layer0.


\end{document}
\grid
\grid




  
